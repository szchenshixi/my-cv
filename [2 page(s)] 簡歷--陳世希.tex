\documentclass{mycv}
%------------------------------------------------------------
%                  Start of Font Adaptation
%------------------------------------------------------------
% [Ref] https://tex.stackexchange.com/questions/483605/creating-a-document-with-mixed-languages
%\usepackage[bidi=default]{babel}
\usepackage[bidi=default]{babel}
\usepackage{fontspec}
\ifluatex
  \defaultfontfeatures{
  	Scale=MatchLowercase,
	Ligatures=TeX,
	Renderer=HarfBuzz
  }
\else
  \defaultfontfeatures{
  	Scale=MatchLowercase,
  	Ligatures=TeX
  }
\fi
%\babelprovide[import, onchar=ids fonts]{chinese-simplified}
\babelprovide[import, onchar=ids fonts]{chinese-traditional}
%\babelfont[chinese-simplified]{rm}
%{Noto Serif SC} % https://fonts.google.com/noto/specimen/Noto+Serif+SC
%\babelfont[chinese-simplified]{sf}
%{Noto Sans SC} % https://fonts.google.com/noto/specimen/Noto+Sans+SC
\babelfont[chinese-traditional]{rm}
{Noto Serif TC} % https://fonts.google.com/noto/specimen/Noto+Serif+TC
\babelfont[chinese-traditional]{sf}
{Noto Sans TC} % https://fonts.google.com/noto/specimen/Noto+Sans+TC
%------------------------------------------------------------
%                    End of Font Adaptation
%----------------------------------------------------------EndEnd\usepackage{tabu}
% Provides easy color coding. e.g. {blue!80!black} means blue*0.8 + black*0.2
\usepackage{xcolor}

\name[陳世希]{陳世希\small{Ph.D. Candidate}}
%\address{Hong Kong University of Science and Technology (HKUST)}
\email{schenax@connect.ust.hk}
\phone{+86-13530088664/+852-67206495}
\homepage{schenax.student.ust.hk}
\github{szchenshixi}
\linkedin{shixi-chen}
%\linkedinId{95a31618b}


\begin{document}
\maketitle

%\section{Research \\ Interests}
%
%I am interested in Databases, Cryptography, Information Security and Privacy-Aware Computing. My current focuses include:
%
%\begin{itemize}
%  \item Authenticated query processing for outsourcing cloud computing.
%  \item Searchable blockchain with integrity assurance.
%  \item Privacy preserving query processing and access control.
%\end{itemize}

\section{職業履歷}

\subsection{香港科技大學 (HKUST)}[]
\begin{positions}
  \entry{博士研究生}{2018年9月~--~現在}
\end{positions}

\begin{itemize}
  \item 導師: \href{https://eexu.home.ece.ust.hk/}{須江教授}
  \item 研究方向包括多核計算架構,片上網絡(NoC)以及軟硬件一體化設計,主要工作包括高性能、低功耗設計和基於集成硅光的新型架構探索
  \item 工作致力於為多核計算系統提供高能效的軟硬件協同的解決方案,為之設計和驗證了數個先進架構、控制模塊及協議
\end{itemize}

\subsection{JADE}[]
\begin{positions}
  \entry{核心開發者}{2019年2月~--~現在}
\end{positions}

\begin{itemize}
  \item \url{https://eexu.home.ece.ust.hk/JADE.html}
  \item 負責開發與維護了一款開源的大型多核系統仿真平台----JADE,並基於此項目展開多核計算架構的探索工作
  \item 設計並實現了多個核心功能單元,包括更好的仲裁器(arbitrator),內存架構(memory hierarchy), optical power delivery systems和片上網絡(NoC)。完成了項目的日常維護和構建工具的升級
\end{itemize}

\subsection{COSMIC}[]
\begin{positions}
	\entry{核心開發者}{2019年2月~--~現在}
\end{positions}

\begin{itemize}
	\item \url{https://eexu.home.ece.ust.hk/COSMIC.html}
	\item 負責開發與維護一套基於統計學模型的開源多核系統基準測試工具(benchmark suite)--COSMIC
	\item 創建並長期維護了5套主流多核測試基準程序,包括APEX, NAS, PARSEC, SPEC, Splash
\end{itemize}

\section{教育經歷}

\subsection{香港科技大學(HKUST)}[]
\vspace{-\parskip}%
\begin{itemize}[label={}]
	\item 電子與計算機工程(Electronic and Computer Engineering) 博士研究生 \printdate{2018年9月~--~現在}
	\begin{itemize}[label=\textbullet]
		\item 以多核系統研究課題為核心,在研究生期間自底向頂完成了CMOS工藝、硅光通信、進階計算機架構、并行算法、優化算法等系列課程
		\item 在包含英語寫作、學術交流等諸多項目中均取得良好成績
		\item CGA: 3.614/4.3
		\item 導師: \href{https://eexu.home.ece.ust.hk/}{須江教授}
	\end{itemize}
\end{itemize}

\subsection{香港科技大學(HKUST)}[]
\vspace{-\parskip}%
\begin{itemize}[label={}]
	\item Bachelor degree of Engineering, First Class Honor (最高榮譽學位)\printdate{2014年9月~--~2018年9月}
	\begin{itemize}[label=\textbullet]
      \item 第一主修為電子與計算機工程(Electronic and Computing Engineering)
      \item 第二主修為計算機科學(Computer Science)
      \item CGA: 3.82/4.3 (<10\% Graduate Students)
	\end{itemize}
\end{itemize}

\section{職業技能}

\begin{description}
	
%  \item[编程] C/C++, Python, Bash, CMake, GNU make, Matlab, Verilog, Java, \LaTeX
  \item[編程]:
  \begin{description}
  	\item[精通] C/C++, Python, Bash, CMake, GNU make
    \item[熟練] Matlab, Verilog, Java, \LaTeX
    \item[基礎] VHDL, SQL, HTML, JavaScript
  \end{description}
  \item[工具] Linux, Vim, Tmux, Git, SVN, Doxygen, Docker % GitBook
  \item[語言] 地道普通話, 流利英語, 基礎日語
\end{description}

\section{榮譽}

%\item Admissions Scholarships (Fall, Winter 2014, Spring, Summer 2015)
%\item University's Scholarship for outstanding academic performance (Fall 2016,
%\item Dean's List (Fall, Spring 2015, Fall, Spring 2016, )

\begin{itemize}
%	\item Dean's List, HKUST \printdate{Fall, Spring 2015\\}
%	\printdate{\hfill Fall, Spring 2016}
%	\item University's Scholarship Scheme, HKUST \printdate{Fall, Winter, Spring, Summer 2014\\}
%	\printdate{\hfill Fall, Winter, Spring, Summer 2016}
%	\item Participant of EDA competition
  \item Teaching Assistant Coordinator, Department of ECE, HKUST \printdate{2018}
  \item University's Scholarship Scheme (共8學期), HKUST\printdate{\hfill 2014~--~2018}
  \item Dean's List (共4學期), HKUST\printdate{\hfill 2015, 2016}
\end{itemize}

\section{發表論文}%

%\textbf{Complete List:}
%\href{https://scholar.google.com/citations?user=DKG_JaAAAAAJ}{Google Scholar \textsf{\footnotesize [DKG\_JaAAAAAJ]}}%
%{~~$\cdot$~~}%
%\href{https://dblp.org/pers/hd/x/Xu_0004:Cheng}{DBLP \textsf{\footnotesize [Xu\_0004:Cheng]}}%

%\publications[keyword={selected}]{publications.bib}

\begin{enumerate}
	\item Jun Feng, Jiaxu Zhang, \textbf{Shixi Chen}, Jiang Xu, ``Scalable Low-Power High-Performance Optical Network for Rack-Scale Computers``, Silicon Photonics for High-Performance Computing and Beyond, CRC 2022.
	\item Fan Jiang, Rafael Kioji Vivas Maeda, Jun Feng, \textbf{Shixi Chen}, Lin Chen, Xiao Li, Jiang Xu, ``Fast and Accurate Statistical Simulation of Shared-Memory Applications on Multicore Systems,`` IEEE Transactions on Parallel and Distributed Systems, 2022.
	\item Xuanqi Chen, Yuxiang Fu, Jun Feng, Jiaxu Zhang, \textbf{Shixi Chen}, Jiang Xu, ``Improving the Thermal Reliability of Photonic Chiplets on Multicore Processors,`` Elsevier Integration, the VLSI Journal, 2022.
	\item Chengeng Li, Fan Jiang, \textbf{Shixi Chen}, Jiaxu Zhang, Yinyi Liu, Yuxiang Fu, Jiang Xu, ``Accelerating Cache Coherence in Manycore Processor through Silicon Photonic Chiplet,`` IEEE/ACM International Conference on Computer Aided Design (ICCAD), 2022.
	\item Yinyi Liu \textsuperscript{\textdagger}, Jiaqi Liu\textsuperscript{\textdagger}, Yuxiang Fu, Shixi Chen, Jiaxu Zhang, Jiang Xu, ``PHANES: ReRAM-based Photonic Accelerators for Deep Neural Networks,`` ACM/IEEE Design Automation Conference (DAC), 2022.
	\item Yinyi Liu, Jiaxu Zhang, Jun Feng, \textbf{Shixi Chen}, Jiang Xu, ``A Reliability Concern on Photonic Neural Networks,`` Design, Automation and Test in Europe Conference and Exhibition (DATE), Antwerp, Belgium, March 2022
	\item Yinyi Liu, Jiaxu Zhang, Jun Feng, \textbf{Shixi Chen}, Jiang Xu, ``Reduce Footprints of Multiport Interferometers by Cosine-Sine-Decomposition Unfolding,`` Optical Fiber Communication Conference and Exhibition (OFC), San Diego, California, USA, March 2022
	\item Jiaxu Zhang, Yingyi Liu, Jun Feng, \textbf{Shixi Chen}, Xiaowen Dong, Wutong Yu, Jiang Xu, ``UONN: Energy-Efficient Optical Neural Network,`` Asia Communications and Photonics Conference (ACP), 2021
	\item \textbf{Shixi Chen}, Jiang Xu, Xuanqi Chen, Zhifei Wang, Jun Feng, Jiaxu Zhang, Zhongyuan Tian, Xiao Li, ``Efficient Optical Power Delivery System for Hybrid Electronic-Photonic Manycore Processors,`` Design, Automation and Test in Europe Conference and Exhibition (DATE), Grenoble, France, March 2020.
	\item Zhifei Wang, Jun Feng, Xuanqi Chen, Zhehui Wang, Jiaxu Zhang, \textbf{Shixi Chen}, Jiang Xu, Systematic Exploration of High-Radix Integrated Silicon Photonic Switches for Datacenters, IEEE/ACM International Conference on Computer Aided Design (ICCAD), Westminster CO, USA, November 2019.
	\item Jun Feng, Zhehui Wang, Zhifei Wang, Xuanqi Chen, \textbf{Shixi Chen}, Jiaxu Zhang, Jiang Xu, ``Scalable Low-Power High-Performance Rack-Scale Optical Network,`` PHOTONICS in SC19, Denver, Colorado, November 2019.
\end{enumerate}

{
\footnotesize %
\textsuperscript{\textdagger}These authors contributed equally.
}

\section{學術服務}

\begin{enumerate}[label={}]
  \item \textbf{Conference Paper Reviewer}
  \begin{itemize}
  	% With dates
%	\item Asia and South Pacific Design Automation Conference (ASP-DAC) \printdate{2020, 2021}
%	\item International Conference on Hardware/Software Codesign and System
%	Synthesis (CODES+ISSS) \printdate{2019, 2020}
%	\item International Conference on Parallel Processing (ICPP) \printdate{2019}
%	\item Computer Society Annual Symposium on VLSI (ISVLSI) \printdate{2019}
%	\item International Symposium on Embedded Multicore/Many-core Systems-on-Chip (MCSoC) \printdate{2019}
%	\item Design Automation Conference (DAC) Ph.D. forum \printdate{2020}
%	\item IEEE/ACM International Symposium on Networks-on-Chip (NOCS) \printdate{2020, 2021}
%	\item Design, Automation and Test in Europe (DATE) \printdate{2020}
%	\item Optical/Photonic Interconnects for Computing Systems (PHOTONIC) \printdate{2019}
	% Without dates
	\item Asia and South Pacific Design Automation Conference (ASP-DAC)
	\item International Conference on Hardware/Software Codesign and System
	Synthesis (CODES+ISSS)
	\item International Conference on Parallel Processing (ICPP)
	\item Computer Society Annual Symposium on VLSI (ISVLSI)
	\item International Symposium on Embedded Multicore/Many-core Systems-on-Chip (MCSoC)
	\item Design Automation Conference (DAC) Ph.D. forum
	\item IEEE/ACM International Symposium on Networks-on-Chip (NOCS)
	\item Design, Automation and Test in Europe (DATE)
	\item Optical/Photonic Interconnects for Computing Systems (PHOTONIC)
	\item Optical Fiber Communication Conference and Exhibition (OFC)
  \end{itemize}
  \item \textbf{Journal Paper Reviewer}
  \begin{itemize}
  	\item IEEE Transactions on Computers (IEEE TC)
  	\item IEEE Transactions on Very Large Scale Integration Systems (TVLSI)
  	\item Transactions on Computer-Aided Design of Integrated Circuits and Systems (TCAD)
  	\item IEEE Transactions on Parallel and Distributed Systems (TPDS)
  \end{itemize}
  \item \textbf{Teaching Assistant}
  \begin{itemize}
%    \item ELEC 2300 Computer Organization, HKUST \printdate{Fall 2019}
%    \item ELEC 4310 Embedded System Design, HKUST \printdate{Spring 2018}
    \item ELEC 2300 Computer Organization, HKUST \printdate{2019}
    \item ELEC 4310 Embedded System Design, HKUST \printdate{2018}
    \item Teaching Assistant Coordinator, Department of ECE, HKUST \printdate{2018}
    \item Participant of EDAthon2020, CEDA \printdate{\hfill 2020}
  \end{itemize}
\end{enumerate}

\end{document}
